The fundamental theorem of invertible matrices

Suppose a is an $n\times n$ matrix, the following statements are equivalent:

$$
\begin{align}
\text{ A is invertible }
\end{align}
$$
$$
\begin{align}
\text{ Solutions + Matrix forms: } \\
a\vec{x}=\vec{b} \text{ has a unique solution for every }\vec{b}\in \mathbb{R}^{n} \\
A\vec{x}=\vec{0} \text{ has only the trivial solution } \\
\text{ The RREF of A is }I_{n} \\
\text{ A is a product of elementary matrices }
\end{align}
$$
$$
\begin{align}
\text{ Columns: } \\
\text{ The column vectors of A are linearly independent } \\
\text{ The column vectors of A span }\mathbb{R}^{n} \\
\text{ The column vectors of A for a basis for } \mathbb{R}^{n}
\end{align}
$$
$$
\begin{align}
\text{ Subspaces: } \\
\text{ Rank(A)} =n \\
\text{ Nullity(A) }=0
\end{align}
$$
$$
\begin{align}
\text{ Rows: } \\
\text{ The row vectors of A are linearly indepenent } \\
\text{ The row vectors of A span } \mathbb{R}^{n} \\
\text{ The row vectors of A form a basis for } \mathbb{R}^{n}
\end{align}
$$
text
